%DO NOT ALTER THIS BLOCK OF COMMANDS.
\documentclass[12pt]{article}
\usepackage[margin=1in, bottom=4.5cm]{geometry}
\usepackage{amsmath,amsthm,amssymb,amsfonts, enumitem, fancyhdr, color, comment, graphicx, environ, scrextend, tikz, mathtools}
\pagestyle{fancy}
\setlength{\headheight}{65pt}
\newenvironment{problem}[2][Problem]{\begin{trivlist}
\item[\hskip \labelsep {\bfseries #1}\hskip \labelsep {\bfseries #2}]}{\end{trivlist}}
\newenvironment{lemma}[2][Lemma]{\begin{trivlist}
\item[\hskip \labelsep {\bfseries #1}\hskip \labelsep {\bfseries #2}]}{\end{trivlist}}
\newenvironment{theorem}[2][Theorem]{\begin{trivlist}
\item[\hskip \labelsep {\bfseries #1}\hskip \labelsep {\bfseries #2}]}{\end{trivlist}}
\newenvironment{proposition}[2][Proposition]{\begin{trivlist}
\item[\hskip \labelsep {\bfseries #1}\hskip \labelsep {\bfseries #2}]}{\end{trivlist}}
\newenvironment{corollary}[2][Corollary]{\begin{trivlist}
\item[\hskip \labelsep {\bfseries #1}\hskip \labelsep {\bfseries #2}]}{\end{trivlist}}
\newenvironment{sol}
    {\emph{Proof.}
    }
    {
    \qed
    }
\specialcomment{com}{ \color{blue} \textbf{Comment:} }{\color{black}} %for instructor comments while grading
\NewEnviron{probscore}{\marginpar{ \color{blue} \tiny Problem Score: \BODY \color{black} }}


\newcommand\restr[2]{{% we make the whole thing an ordinary symbol
  \left.\kern-\nulldelimiterspace % automatically resize the bar with \right
  #1 % the function
  \vphantom{\big|} % pretend it's a little taller at normal size
  \right|_{#2} % this is the delimiter
  }}
%%%%%%%%%%%%%%%%%%%%%%%%%%%%%%%%%%%%%%%

%%%%%%%%%%%%%%%%%%%%%%%%%%%%%%%%%%%%%%%%%%%%%
%Fill in the appropriate header information below
\lhead{Trey Manuszak \\ Christian Garcia}  %replace with your name
\rhead{MAT 445: Number Theory \\ 3/30 Lecture Notes}
%%%%%%%%%%%%%%%%%%%%%%%%%%%%%%%%%%%%%%%%%%%%%

%%%%%%%%%%%%%% PUT YOUR TITLE PAGE INFO HERE
\usepackage{blindtext}
\title{MAT 445: Number Theory}
\date{April 7, 2020}
\author{Trey Manuszak\\ Arizona State University}
%%%%%%%%%%%%%%



%%%%%%%%%%%%%%%%%%%%%%%%%%%%%%%%%%%%%%
%Do not alter this block.
\begin{document}


%%%%%%%%%%%%%%%%%%%%%%%%%%%%%%%%%%%%

%%%%%THIS IS WHERE YOU WILL BE DOING ALL OF YOUR WORK
%Copy the following block of text for each problem in the assignment.
\section{Introduction and Review}

\section{Wilson's Theorem and Consequences}

\section{Euler's Criterion for Squares (mod \textit{\textbf{p}})}

\begin{theorem}{8.5 (Euler's Criterion for Squares (mod \textit{\textbf{p}}))} \textit{Let $p$
be an odd prime number, and let $a$ be an integer coprime to $p$. Then,
\begin{itemize}
  \item $a$ is square$\pmod{p}$, if and only if $a^{(p-1)/2} \equiv 1 \pmod{p}$.
  \item $a$ is nonsquare$\pmod{p}$, if and only if $a^{(p-1)/2} \equiv -1 \pmod{p}$.
\end{itemize}}
\end{theorem}

\section{Applications of Euler's Criterion}

\begin{theorem}{8.7 (Fermat's Christmas Theorem)}
  \textit{Let $p$ be a prime number with $p \equiv 1 \pmod{4}$. Then the Diophantine equation $x^2 + y^2 = p$ has a solution. In other words, $p$ can be expressed as the sum of two squares.}
\end{theorem}

\begin{proposition}{8.8 (Minkowski's theorem in the plane)} \textit{Consider a grid of parallelograms in the plane, with the origin at a grid point, and a circle centered at the origin. If the area of the circle is greater than 4 times the area of the parallelogram, then the circle contains a grid-point besides the origin.}
  
\end{proposition}


%%%%%%%%%%%%%%%%%%%%%%%%%%%%%%%%%%%%%%%%
%Do not alter anything below this line.
\end{document}